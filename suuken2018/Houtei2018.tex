\documentclass[11pt, a4paper, dvipdfmx]{jsarticle}
    \usepackage{amsmath}
    \usepackage{amsthm}
    \usepackage[psamsfonts]{amssymb}
    \usepackage{color}
    \usepackage{ascmac}
    \usepackage{amsfonts}
    \usepackage{mathrsfs}
    \usepackage{amssymb}
    \usepackage{graphicx}
    \usepackage{fancybox}
    \usepackage{enumerate}
    \usepackage{verbatim}
    \usepackage{subfigure}
    \usepackage{proof}
    \usepackage{listings}
    \usepackage{otf}
 %
    \theoremstyle{definition}
    %
    %%%%%%%%%%%%%%%%%%%%%%%%%%%%%%%%%%%%%%
    %ここにないパッケージを入れる人は,必ずここに記載すること.
    %
    %%%%%%%%%%%%%%%%%%%%%%%%%%%%%%%%%%%%%%
    %ここからはコード表です.
    %
   \newtheorem{Axiom}{公理}[section]
    \newtheorem{Definition}[Axiom]{定義}
    \newtheorem{Theorem}[Axiom]{定理}
    \newtheorem{Proposition}[Axiom]{命題}
    \newtheorem{Lemma}[Axiom]{補題}
    \newtheorem{Corollary}[Axiom]{系}
    \newtheorem{Example}[Axiom]{例}
    \newtheorem{Claim}[Axiom]{主張}
    \newtheorem{Property}[Axiom]{性質}
    \newtheorem{Attention}[Axiom]{注意}
    \newtheorem{Question}[Axiom]{問}
    \newtheorem{Problem}[Axiom]{問題}
    \newtheorem{Consideration}[Axiom]{考察}
    \newtheorem{Alert}[Axiom]{警告}
    %%%%%%%%%%%%%%%%%%%%%%%%%%%%%%%%%%%%%%
    %
    %定義や定理等に番号をつけたくない場合(例えば定理1.1等)は以下のコードを使ってください.
    %但し,例えば\Axiom*{}としてしまうと番号が付いてしまうので,必ず \begin{Axiom*} \end{Axiom*}の形で使ってください.
    \newtheorem*{Axiom*}{公理}
    \newtheorem*{Definition*}{定義}
    \newtheorem*{Theorem*}{定理}
    \newtheorem*{Proposition*}{命題}
    \newtheorem*{Lemma*}{補題}
    \newtheorem*{Example*}{例}
    \newtheorem*{Corollary*}{系}
    \newtheorem*{Claim*}{主張}
    \newtheorem*{Property*}{性質}
    \newtheorem*{Attention*}{注意}
    \newtheorem*{Question*}{問}
    \newtheorem*{Problem*}{問題}
    \newtheorem*{Consideration*}{考察}
    \newtheorem*{Alert*}{警告}
    \renewcommand{\proofname}{\bfseries Proof}
    
    \newcommand{\A}{\bf 証明}
    \newcommand{\B}{\it Proof}
    
    %
    %%%%%%%%%%%%%%%%%%%%%%%%%%%%%%%%%%%%%%
    %英語で定義や定理を書きたい場合こっちのコードを使うこと.
    \newtheorem{Axiom+}{Axiom}[section]
    \newtheorem{Definition+}[Axiom+]{Definition}
    \newtheorem{Theorem+}[Axiom+]{Theorem}
    \newtheorem{Proposition+}[Axiom+]{Proposition}
    \newtheorem{Lemma+}[Axiom+]{Lemma}
    \newtheorem{Example+}[Axiom+]{Example}
    \newtheorem{Corollary+}[Axiom+]{Corollary}
    \newtheorem{Claim+}[Axiom+]{Claim}
    \newtheorem{Property+}[Axiom+]{Property}
    \newtheorem{Attention+}[Axiom+]{Attention}
    \newtheorem{Question+}[Axiom+]{Question}
    \newtheorem{Problem+}[Axiom+]{Problem}
    \newtheorem{Consideration+}[Axiom+]{Consideration}
    \newtheorem{Alert+}{Alert}
    %\renewcommand{\proofname}{\bfseries 証明}
    %
    %
    %%%%%%%%%%%%%%%%%%%%%%%%%%%%%%%%%%%%%%
    %数
    \newcommand{\N}{\mathbb{N}}
\newcommand{\Z}{\mathbb{Z}}
\newcommand{\R}{\mathbb{R}}
\newcommand{\C}{\mathbb{C}}
\newcommand{\W}{{\cal W}}
\newcommand{\cS}{{\cal S}}
\newcommand{\Wpm}{W^{\pm}}
\newcommand{\Wp}{W^{+}}
\newcommand{\Wm}{W^{-}}
\newcommand{\p}{\partial}
\newcommand{\Dx}{D_{x}}
\newcommand{\Dxi}{D_{\xi}}
\newcommand{\lan}{\langle}
\newcommand{\ran}{\rangle}
\newcommand{\pal}{\parallel}
\newcommand{\dip}{\displaystyle }
\newcommand{\e}{\varepsilon}
\newcommand{\dl}{\delta}
\newcommand{\pphi}{\varphi}
\newcommand{\ti}{\tilde}
    \title{Submodular function maximization and Greedy algorithm}
    \author{yataka}
    %\date{}
\begin{document}
\maketitle
%ソースコードの設定
\lstset{
  basicstyle={\ttfamily},
  identifierstyle={\small},
  commentstyle={\smallitshape},
  keywordstyle={\small\bfseries},
  ndkeywordstyle={\small},
  stringstyle={\small\ttfamily},
  frame={tb},
  breaklines=true,
  columns=[l]{fullflexible},
  numbers=left,
  xrightmargin=0zw,
  xleftmargin=3zw,
  numberstyle={\scriptsize},
  stepnumber=1,
  numbersep=1zw,
  lineskip=-0.5ex
}
\section{はじめに}
この方程では$V$は有限集合であるとする.
\section{劣モジュラ関数}
まず劣モジュラ関数を定義する.
\begin{Definition+}(Discreate derivative)\\
    $f$を集合関数とする. $S$を$V$の部分集合とし$e$を$V$の元とする. この時$f$の$S$での
    Discreate derivativeを
    \begin{align*}
        \Delta_{f}(e | S) := f(S\cup\{e\}) - f(S)
    \end{align*}
    と定義する.
\end{Definition+}
\begin{Definition+}(Submodular function)\\
    $f$を集合関数とする. $f$がSubmodular functionであるとは
    \begin{align*}
        \forall A, B\subset V, \forall e\in V,A\subset B かつe\in V - B\Longrightarrow \Delta_{f}(e | A)\leq\Delta_{f}(e | B)
    \end{align*}
    を満たすことをいう.
\end{Definition+}
\begin{Theorem+}$f$を集合関数とする. この時
    \begin{enumerate}
        \item $f$は Submodular function である.
        \item $\forall A, B\subset V, f(A\cap B) + f(A\cup B)\leq f(A) + f(B)$ 
    \end{enumerate}
    は同値である.
\end{Theorem+}
まあ, ここまでよくわからない Submodular function というものを定義して来たが, 読者の中には
「$V$は有限集合なんだから, $V$の要素を入っているか入っていないかで分けて全通り試せば良い.」 と思う人もいるかもしれない.
実際にそれをC++で実装すると以下のようになる.
\begin{lstlisting}[caption=C++のソースコード,label=ほげ]
#include<iostream>
#include<math.h>
using namespace std;
#define MAX_N 10
int main(){
    int f_V[MAX_N];
    for(int i = 0;i < MAX_N; i++){
        if(i % 3 ==0){
            f_V[i] = -2 * i + 1; 
        }else{
            f_V[i] = 2 * i + 1; 
        }
    }
    int sum[1024] = {0};
    for(int i = 0; i < pow(2, MAX_N); i++){
        for(int k = 0; k < MAX_N; k++){
            if(i & 1<<k){
                sum[i] += f_V[k];
            }
        }
    }
    for(int i = 0; i < 1024; i++){
        cout << sum[i] << " ";
    }
    cout << endl;
    return 0;
}
 \end{lstlisting}
 このコードでは, $|V| = 10$の場合を考えて実装している. このように$|V|$が小さい場合であればこのような方法でも
 問題はない. しかしながら, $|V| = 10000$の場合はどうだろうか? 調べる候補の数は$2^{10}$から$2^{10000}$へと
 跳ね上がる. この量を全て調べ上げるのは現実的に不可能である.(100年とか待てるなら別ですが...)\\
 というわけで, 集合関数$f$が最大となる$V$の部分集合$S$を効率に求めることが必要なのである.
\end{document}
